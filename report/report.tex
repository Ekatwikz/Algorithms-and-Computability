\documentclass[a4paper]{article}
\usepackage{amssymb}
\usepackage{amsmath}

%\usepackage[T2A,T1]{fontenc}
%\usepackage[utf8]{inputenc}
%\usepackage[russian,english]{babel}

\usepackage{algorithm}
\usepackage{algpseudocode}

\usepackage{hyperref}

\usepackage{graphicx}
\graphicspath{{images/}}

\usepackage[
backend=bibtex,
style=alphabetic,
sorting=ynt
]{biblatex}
\addbibresource{citations.bib}

\usepackage{amsthm}
\newtheorem{proposition}{Proposition}
\newtheorem{lemma}{Lemma}

%%% Title Page Stuff:
\title{Graph Metric, Max Common Subgraph and Max Clique Algorithms}
\author{Dipankar Purecha, Emmanuel Katwikirize, \\ Ibrahim Abu Alshayeb, Yuldashbek Yusupov}
\date{\today}
\newcommand{\course}{Algorithms and Computability}

\makeatletter
\let\thetitle\@title
\let\theauthor\@author
\let\thedate\@date
\makeatother

\usepackage{fancyhdr}
\pagestyle{fancy}
\fancyhf{}
\lhead{\thetitle}
\cfoot{\thepage}
%%% End of title page stuffz

% https://tex.stackexchange.com/a/40924/6832
\newcommand\inputfile[1]{%
	\InputIfFileExists{#1}{}{\typeout{No file "#1"??}}%
}

\begin{document}
\inputfile{sections/title}

\tableofcontents
\cleardoublepage % Zaczynamy od nieparzystej strony
\pagestyle{headings}

\section{Metric}
Our main observation was that there are many functions $d'$ that are intuitively ``almost metrics'' for graphs,\\
ie: they have all the properties of a metric, except for the guarantee that 0 distance implies equality. \\
We aim to take such a $d'$ and create a metric $d$ by adjusting it. \\
In case it is required that isomorphic graphs have 0 distance to each other, we also explain how we can instead create a metric on the space of classes of isomorphic graphs. \\
This approach allows us to use $d'$ as a basic but fast approximation for $d$ as well.

\subsection{The method}
	We take a set $S$ and a function $d'\colon S \times S \mapsto \mathbb{N} \cup \{0\}$ \\
	Such that $\forall \{a, b, c\} \subseteq S$: \\
	\begin{equation} \label{statement:1}
		a = b \implies d'(a, b)=0
	\end{equation}
	\begin{equation} \label{statement:2}
		d'(a, b) = d'(b, a)
	\end{equation}
	\begin{equation} \label{statement:3}
		d'(a, c) \leq d'(a, b) + d'(b, c)
	\end{equation}
	We can also create a relation:
	\[a\sim_{d'}b \iff d'(a, b)=0\]
	
	\begin{proposition}
		$\sim_{d'}$ is an equivalence relation
	\end{proposition}
	\begin{proof} $\forall \{a, b, c\} \subseteq S$: \\
		(\ref{statement:1}) $\implies a \sim a$, \\
		(\ref{statement:2}) $\implies a \sim b \iff b \sim a$, \\
		$(\ref{statement:3}) \land a \sim b \land b \sim c
		\implies d'(a, c) \leq d'(a,b) + d'(b, c) = 0 + 0 \implies a \sim c $ \\
	\end{proof}
	
	We will use the usual notation of equivalence classes where \\
	$[a]_{\sim} = \{x \mid x \sim a \}$, and for convenience just write $[a] = \{x \mid x \sim_{d'} a\}$

\newpage

	\begin{proposition}
		\[ d(a, b) = \begin{cases}
			1 & d'(a,b)=0 \land a \neq b \\
			d'(a,b) & otherwise
		\end{cases}
		\] is a metric on S
	\end{proposition}
	\begin{proof} $\forall \{a, b, c\} \subseteq S$:
		\begin{equation} \label{statement:4}
			a = b \iff d(a, b)=0
		\end{equation}
		By definition, $a=b \implies d(a, b) = d'(a, b) = 0$ \\
		For the converse, $d(a, b) = 0 \implies $ \\
		$ d(a, b) \neq 1 \land ( d(a, b) = d'(a, b) \neq 0 \lor a=b) \implies $ \\
		$ a = b $
		
		\begin{equation} \label{statement:5}
			d(a, b) = d(b, a)
		\end{equation}
		If $d(a, b) \neq 1$: \\
		$d(a, b) = d'(a, b) = d'(b, a) = d(a, b) $ \\
		Otherwise: \\
		$d'(a,b)=0 \land a \neq b \iff d'(b,a)=0 \land b \neq a $ \\
		so: \\
		$d(a, b)=1 \iff d(b, a)=1$
		
		\begin{equation} \label{statement:6}
			d(a, c) \leq d(a, b) + d(b, c)
		\end{equation}
		\begin{lemma}
		\label{lemma1}
			$[a]=[b] \implies d'(a,c) = d'(b, c)$
		\end{lemma}
		\begin{proof} Assume (for the sake of contradiction), that (WLOG) \\
			$[a]=[b] \land  d'(a , c)< d'(b , c)$ \\
			Then $[a]=[b] \iff d'(a, b)=d'(b, a)=0$ (by definition) \\
			and $d'(b, c) \leq d'(b, a) + d'(a, c)$ (by (\ref{statement:3})) \\
			$ \implies d'(b, c) \leq 0 + d'(a, c) $ which contradicts the initial assumption
		\end{proof}
		\begin{lemma}
		\label{lemma2}
			$[a] \neq [b] \implies d(a, b) = d'(a, b) > 0$
		\end{lemma}
		\begin{lemma}
		\label{lemma3}
			$[a]=[b] \implies d(a, b)=d_1(a , b)$ \\
			Where $d_1$ is the discrete metric
		\end{lemma}
		Lemma \ref{lemma2} and \ref{lemma3} follow directly from the definitions.\\
		So, proceeding with (\ref{statement:6}):\\
		
		If $[a]=[b]=[c]$, Lemma \ref{lemma3} implies (\ref{statement:6}):
		\begin{proof}
			We know that $d=d_1$ is a metric for all pairs in this case
		\end{proof}
		
		If $[a]=[b]\land[a]\neq[c ]\land[b]\neq[c ]$, Lemma \ref{lemma1}, \ref{lemma2} and \ref{lemma3} imply (\ref{statement:6}):
		\begin{proof}
			We know that $d (a , b) = d_1(a,b) \leq 1 \land d (a , c)=d (b , c)=d ' (a , c)>0$
		\end{proof}
		
		If $[a]$, $[b]$ and $[c]$ are pairwise different, Lemma \ref{lemma2} implies (\ref{statement:6}):
		\begin{proof}
			Assume (for the sake of contradiction) that Lemma \ref{lemma2} is true but (\ref{statement:6}) is false for this case, then Lemma \ref{lemma2} implies that (\ref{statement:3}) does not hold for $d'$
		\end{proof}
		
		So (\ref{statement:4}), (\ref{statement:5}) and (\ref{statement:6}) hold, hence $d$ is a metric.
	\end{proof}

\newpage	
\subsection{Examples of $d'$ for graphs}
	We can then take $S$ as the set of all graphs, and pick an appropriate $d'$ to
	“extend” to a $d$ in this way. \\
	If we require isomorphic graphs to have 0 distance to each other, then we can
	instead interpret $S_{\cong}$ as being the class of sets $[g]_{\cong}$ where $g_1 \cong g_2 \iff g_1$ is
	isomorphic to $g_2$ , and use abuse of notation to write the metrics on $S_{\cong}$ with
	the same syntax as those on $S$ \\
	(one way is by relating $g$ to $[g]_{\cong}$ , and making sure
	that any $f(g)$ used in the definition of a $d'$ on $S$ has the property $g_1 \cong g_2 \implies f (g_1)=f (g_2)$ ). \\
	
	We say that $A(f)$ is an algorithm that computes $f$. \\
	In the $S_{\cong}$ case, any $A(d)$ has a time complexity of at least $2^{poly(\log n)}$ \cite{DBLP:journals/corr/Babai15} due to the existence of a reduction from the GI problem to the computation of $d$. \\
	So, any $A(d')$ can be used to approximate its respective $d$, which will always be faster
	than $A(d)$ and exact whenever $[g_1]\neq[g_2]$.\\
	
	Some simple functions which have the properties of $d'$ are:
	\begin{itemize}
		\item $d'(g_1 , g_2)=d_{\mathbb{N} \cup \{0\}} (|g_1|,|g_2|)$ \\
		where $|g|$ is the size of $g$ and \\
		and $d_{\mathbb{N} \cup \{0\}}$ is any metric on $\mathbb{N} \cup \{0\}$ \\
		
		Our program (Algorithm \ref{alg:metric1}) will cover this case, with:\\
		$d_{\mathbb{N} \cup \{0\}}(a, b) = |a - b|$ \\
		and $|g|$ = $|V| + |E|$, where $V$ is the vertex set of the graph, and $E$ is the edge set of the graph
		
		\item $d'(g_1, g_2)=d_M (degrees(g_1), degrees(g_2))$ \\
		Where $degrees(g)$ is the multiset of vertex degrees in $g$, \\
		and $d_M$ is any metric on multisets, such as $d_M = d_1$
		
		\item $d'(g_1 , g_2)=d_{str} (str ( p_{g_1} (x)), str ( p_{g_2} (x)))$ \\
		where $p_g(x)$ is the characteristic polynomial of $g$ \\
		$str$ is any function that injectively maps polynomials to strings \\
		and $d_{str}$ is any metric on strings
		
		an example of $str$ would be to take the coefficients of the polynomial in ascending order, and write them separated with spaces \\
		an example of $d_{str}$ would be Levenshtein Distance
	\end{itemize}
	
\newpage
	
\subsection{The algorithm}
\begin{algorithm}[H]
	\caption{Our metric for graphs}\label{alg:metric1}
	\begin{algorithmic}
		\Procedure {MetricDistance}{$M_{g_1}$, $M_{g_2}$, $justApprox$}
			\State $d \gets |GraphSize(M_{g_1}) - GraphSize(M_{g_2})|$
			
			\If{$d > 0 \lor justApprox$}
			\State \textbf{return $d$}
			\ElsIf{$IsIsomorphicTo(M_{g_1}, M_{g_2})$}
			\State \textbf{return $0$}
			\Else
			\State \textbf{return $1$}
			\EndIf
		\EndProcedure
		
		\\
		
		\Procedure {GraphSize}{$M$}
			\State $n \gets VertexCount(M)$
			\State $vertsAndEdges \gets n$
			
			\\
			
			\For{$i \gets 0, n - 1$}
			\For{$j \gets 0, n - 1$}
			\State $vertsAndEdges \gets vertsAndEdges + M_{i, j}$
			\EndFor
			\EndFor
			\State \textbf{return $vertsAndEdges$}
		\EndProcedure
		
		\\
		
		\Procedure {IsIsomorphicTo}{$M_{g_1}$, $M_{g_2}$}
			\If{$GraphSize(M_{g_1}) \neq GraphSize(M_{g_2})$}
				\State \textbf{return $false$}
			\EndIf
		
			\\
				
			\State $n \gets VertexCount(M_{g_1})$
			
			\\
			
			\Procedure {IsPermutationIsomorphism}{$A$, $B$, $p$}
				\For{$lhsPos \gets 0, n - 1$}
					\For{$rhsPos \gets 0, n - 1$}
						\If{$A_{lhsPos, rhsPos} \neq B_{p[lhsPos], p[rhsPos]}$}
							\State \textbf{return $false$}
						\EndIf
					\EndFor
				\EndFor
				\State \textbf{return $true$}
			\EndProcedure
			
			\\
			
			\State $iota \gets Iota(n)$ \Comment{array of n ascending values, starting from 0}
			\State $perms \gets HeapsPerms(iota)$
			
			\For{$i \gets 0,n!-1$}
				\If{$IsPermutationIsomorphism(M_{g_1}, M_{g_2}, perms[i])$}
					\State \textbf{return $true$}
				\EndIf
			\EndFor
			\State \textbf{return $false$}
		\EndProcedure
		
		\algstore{bkbreak}
	\end{algorithmic}
\end{algorithm}
\begin{algorithm}[h]
	\begin{algorithmic}[1]
		\algrestore{bkbreak}
		\Procedure {HeapsPerms}{$array$} \Comment{array of all perms of array\cite{10.1093/comjnl/6.3.293}}
			\State $n \gets length(array)$
			\State $c \gets Iota(n)$ \Comment{stack state encoding}
			\State $outputArray \gets []$
			\State $outputArray \gets outputArray \cup array$
			
			\\
			
			\State $i \gets 1$
			
			\While{$i < n$}
				\If{$c[i] < i$}
					\If{$i$ is even}
						\State $Swap(A[0], A[i])$
					\Else
						\State $Swap(A[c[i]], A[i])$
					\EndIf
					
					\\
					
					\State $outputArray \gets outputArray \cup array$
					\State $c[i] \gets c[i] + 1$
					\State $i \gets 1$
				\Else
					\State $c[i] \gets 0$
					\State $i \gets i + 1$
				\EndIf
			\EndWhile
			
			\\
			
			\State \textbf{return $outputArray$}
		\EndProcedure		
	\end{algorithmic}
\end{algorithm}

\newpage

\subsection{Time Complexity}
The $IsPermutationIsomorphism$ procedure is $O(n^2)$ \\
The $HeapsPerms$ procedure is $O(n!)$ \cite{10.1093/comjnl/6.3.293}, and generates a set with $n!$ elements \\
Therefore, the $IsIsomorphicTo$ procedure is $O(n^2 n!)$, since its most expensive step is at the end, where in the worst case it calls $IsPermutationIsomorphism$ on each the element generated by $HeapsPerms$ \\
Assuming $VertexCount$ is $O(1)$ \cite{cppstandard}, the $GraphSize$ procedure will be $O(n^2)$

We can then conclude that if $justApprox$ is false, the worst case time complexity of $MetricDistance$ will $O(n^2 n!)$, otherwise it will be $O(n^2)$

\section{Max Subgraph}
We decided to go with Maximum Induced Subgraph as our Max Subgraph method. The main priority is given to a subgraph with higher number of vertices. In case, several such graphs are found, the priority is given to the one with the highest number of connections and then to the one with higher number of edges.

\subsection{The Method}
For the method of finding Max Subgraph, we decided to implement the Modular Product of two graphs and run the Max Clique method on it, with the small changes made for both of these methods. The changes were required to work with the multi-graphs:
\begin{enumerate}
    \item In the Max Clique method [\ref{alg:modmaxclique}], one way connections are also considered, so that the Max Subgraph will not lack any one way connected vertices.
    \item In the Modular Product graph, the minimum of weights of vertices from provided graphs are taken.
\end{enumerate}

\subsubsection{The modular product}
The Modular  Product of graphs $G$ and $H$ is a graph formed by combining $G$ and $H$. The vertex set of the modular product of $G$ and $H$ is the cartesian product {$V(G)$} × {$V(H)$}. Any two vertices {$(u, v)$} and {$(u' , v' )$} are adjacent in the modular product of $G$ and $H$ if and only if $u$ is distinct from $u'$, $v$ is distinct from $v'$, and either:
\begin{itemize}
    \item $u$ is adjacent with $u'$ and $v$ is adjacent with $v'$, or
    \item $u$ is not adjacent with $u'$ and $v$ is not adjacent with $v'$.
\end{itemize}
The ${weight(uv, u'v')}$ is equal to either:
\begin{itemize}
    \item ${min(weight(u, u'), weight(v, v'))}$ if the first condition of the above statement is hold, or
    \item $1$, otherwise.
\end{itemize}


\subsubsection{The Max Subgraph}
As you may have noticed, The Modular Product of graphs $G$ and $H$ does not contain self loops, because of how vertex adjacency was defined for it. Therefore, additional logic has to be implemented to include the self looped vertices.\newline
Knowing the vertices that are given by The Max Clique method, it is easy to recover the original vertices from both provided graphs, since the adjacency matrix for The Modular Product graph is constructed the following way:\newline
Given $V(G)=\{G_1, G_2, G_3, ...\}$ and $V(H)=\{H_1, H_2, H_3, ...\}$, the adjacency matrix is as follows
\begin{center}
\begin{tabular}{c|c c c c c c c}
              & $G_1H_1$ & $G_1H_2$ & $G_1H_3$ & \dots & $G_2H_1$ & $G_2H_2$ & \dots \\
    \hline
     $G_1H_1$ &  \\
     $G_1H_2$ &  \\
     $G_1H_3$ &  \\
     \vdots   &  \\
     $G_2H_1$ &  \\
     $G_2H_2$ &  \\
     \vdots   & 
\end{tabular}
\end{center}
Therefore, to retrieve the original vertices of graph $G$, we simply need to divide by $|V(H)|$, whereas to retrieve the original vertices of graph $H$, we need to modulo by $|V(H)|$.

\newpage
\subsection{The Algorithm}
\begin{algorithm}
    \caption{Our Modular Product}\label{alg:modprod}
    \begin{algorithmic}
		\Procedure{ModularProduct}{$G$, $H$}
            \State $n \gets |V(G)| * |V(H)|$
            \State $resultMat \gets [][]$ \Comment{Size of the matrix is $n \times n$} \\
    
            \For{$row \gets 0, n-1$}
                \For{$col \gets 0, n-1$}
                    \State $Grow \gets row / |V(H)|$
                    \State $Gcol \gets col / |V(H)|$
                    \State $Hrow \gets row \pmod {|V(H)|}$
                    \State $Hcol \gets col \pmod {|V(H)|}$
                    \\
                    \If{$row == col \lor Grow == Gcol \lor Hrow == Hcol$}
                        \State $continue$
                    \EndIf
                    \\
                    \State $resultMat[row][col] \gets min(G[Grow][Gcol], H[Hrow][Hcol])$
                    \\
                    \If{$G[Grow][Gcol] == 0 \land H[Hrow][Hcol] == 0$}
                        \State $resultMat[row][col] \gets 1$
                    \EndIf
                \EndFor
            \EndFor
            \\
            \State \textbf{return $resultMat$}
            \\
        \EndProcedure
    \end{algorithmic}
\end{algorithm}

\newpage

\begin{algorithm}
    \caption{Our Max Subgraph}\label{alg:maxsubgraph}
    \begin{algorithmic}
		\Procedure{MaxSubgraph}{$G$, $H$, $justApprox$}
            \State $modProd \gets ModularProduct(G, H)$
            \State $maxClique \gets ModifiedMaxClique(modProd, justApprox)$
            \State $n \gets |V(maxClique)|$
            \State $resultMat \gets [][]$ \Comment{Size of the matrix is $n \times n$} 
            \State $Gverts \gets []$ \Comment{Size of the array is $n$}
            \State $Hverts \gets []$ \Comment{Size of the array is $n$}
            \\
            \For{$i \gets 0, n-1$}
                \State $Gverts[i] \gets maxClique[i] / |V(H)|$
                \State $Hverts[i] \gets maxClique[i] \pmod {|V(H)|}$
            \EndFor
    
            \For{$row \gets 0, n-1$}
                \For{$col \gets 0, n-1$}
                    \State $resultMat[row][col] \gets min(G[Grow][Gcol], H[Hrow][Hcol])$
                \EndFor
            \EndFor
            \\
            \State \textbf{return $resultMat$}
            \\
        \EndProcedure
    \end{algorithmic}
\end{algorithm}

\newpage

\subsection{Time and Space complexity}
The time complexity of the $Modular Product$ method is $O((n*m)^2)$, where $n$ is the number of vertices of the first graph and $m$ is the number of vertices of the second graph. \\
The space complexity of the $Modular Product$ method is $O(G + H + (n*m)^2)$, where $n$ is the number of vertices of G and $m$ is the number of vertices of H. Assuming that the most space is taken by the adjacency matrix in the graphs G and H, we can write the space complexity as $O((n*m)^2)$.\\
The time complexity of the $Max Subgaph$ method is dependant on the $Modified Max Clique$ method and $Modular Product$ method. In case no $justApprox$ value is passed or in case $justApprox$ is equal to exact, the time complexity is $O(n*2^n)$, due to $Modified Max Clique$ time complexity being $O(2^n)$. However, if approximation is used instead, which means the time complexity of the $Modified Max Clique$ method goes to $O(n^3)$, the time complexity of $Max Subgraph$ method will be $O((n*m)^2)$, where $n$ is the number of vertices of the first graph and $m$ is the number of vertices of the second graph.\\
The space complexity of the $Max Subgraph$ method is $O((n*m)^2)$, for the same reason as in $Modular Product$.
\section{Max Clique}
A clique, \(C\), in a graph \(G = (V, E)\) is a subset of vertices in a graph, \(C \subseteq V\), such that every pair of distinct vertices are adjacent, i.e., each vertex in the pair has a directed edge to the other. The maximum clique is the clique containing the most number of vertices. This algorithm for the maximum clique has been kept the same for single and multi-edged graphs due to the way we define the maximum clique.
\subsection{Description of algorithm}
We use a simple recursive backtracking algorithm for finding the maximum clique of a graph. It initially starts with an empty list denoting the current clique. It proceeds to check if the current node being examined is adjacent to all the other nodes and if so, it adds it to the current clique set and then recursively calls the same function for the next node. It does this for all the nodes and hence checks all the different combinations of the nodes to see if it forms the maximum clique.
\subsection{Time and Space Complexity of algorithm}
The algorithm's time complexity is $O(n*2^n)$ since it recursively explores all possible cliques of the graph. The space complexity is $O(n)$ if we only store only one set of vertices and $O(2^n)$ if we store all the vertices forming the maximum clique.

Since this is an exponential time algorithm, we create a polynomial time approximation for it by stopping the algorithm after $O(n^2)$ executions of the recursive function.

\newpage

\subsection{Algorithm}
    \begin{algorithm}
    \caption{Max Clique Algorithm}
    \begin{algorithmic}[1]
    \Procedure{MaxClique}{G, justApprox}
        \State $\text{currentClique} \gets \emptyset$
        \State $\text{maxCliques} \gets \emptyset$
        \If{\text{justApprox}}
            \State $\text{maxExecutions} \gets |G|^2$
        \EndIf
        \State \Call{MaxCliqueHelper}{0, \text{currentClique}, \text{maxCliques}, \text{justApprox}}
        \State \textbf{return} \text{maxClique}
    \EndProcedure
    
    \\
    
    \Procedure{MaxCliqueHelper}{curVertex, curClique, maxCliques, justApprox}
        \If{$|\text{currentClique}| > |\text{maxCliques[0]}|$}
            \State $\text{maxCliques} \gets \{\text{curClique}\}$
        \EndIf
        \If{$|\text{currentClique}| == |\text{maxCliques[0]}|$}
            \State $\text{maxCliques}\text{.insert(curClique)}$
        \EndIf
    
        \If{\text{curVertex} == \text{vertexCount}}
            \State \textbf{return}
        \EndIf
        
        \If{\text{justApprox}}
            \If{++currentExecution >= maxExecutions}
                \State \textbf{return}
            \EndIf
        \EndIf
    
        \For{$i \gets \text{curVertex}$ \textbf{to} \text{vertexCount} \textbf{step} 1}
            \If{\Call{IsAdjacentToAllNodesInClique}{$i, \text{curClique}$}}
                \State $\text{currentClique}.\text{push\_back}(i)$
                \State \Call{MaxCliqueHelper}{$i + 1, \text{curClique}, \text{maxCliques}$}
                \State $\text{curClique}.\text{pop\_back}()$
            \EndIf
        \EndFor
    \EndProcedure
    \end{algorithmic}
    \end{algorithm}


\subsection{Modified Max Clique for Max Induced Subgraph}
The algorithm presented below is designed for finding max induced subgraph. It leverages the same helper method, albeit with a different definition of adjacency. In this context, adjacency is determined by checking if there is at least a single-sided edge between the vertices in the clique, as opposed to a double-sided edge. Once it has found all the maximum cliques, it returns the maximum clique which has the highest number of double sided edges and then then highest number of total edges. 

\begin{algorithm}
    \caption{Modified Max Clique For Induced Subgraph}\label{alg:modmaxclique}
    \begin{algorithmic}[1]
    \Procedure{ModifiedMaxClique}{$\text{G}, \text{justApprox}$}
        \State $\text{currentClique} \gets \emptyset$
        \State $\text{maxCliques} \gets \{\{\}\}$
        \If{\text{justApprox}}
            \State $\text{maxExecutions} \gets |G|^2$
        \EndIf
    
        \State $\text{MaxCliqueHelper}(0, \text{currentClique}, \text{maxCliques}, \text{accuracy})$
    
        \State \textbf{return} $\text{CliqueWithMaxSize}(\text{maxCliques})$
    \EndProcedure
    \\
    \Procedure{EdgeCount}{$\text{clique}$}
        \State $\text{edgeCount} \gets 0$
        \For{$i \gets 0$ \textbf{to} $\text{clique.size()} - 1$}
            \For{$j \gets i + 1$ \textbf{to} $\text{clique.size()} - 1$}
                \State $\text{edgeCount} \gets \text{edgeCount} + \text{adjacencyMatrix}[\text{clique}[i]][\text{clique}[j]] + \text{adjacencyMatrix}[\text{clique}[j]][\text{clique}[i]]$
            \EndFor
        \EndFor
        \State \textbf{return} $\text{edgeCount}$
    \EndProcedure
    \\
    \Procedure{TotalConnections}{$\text{clique}$}
        \State $\text{totalWeight} \gets 0$
        \For{$i \gets 0$ \textbf{to} $\text{clique.size()} - 1$}
            \For{$j \gets i + 1$ \textbf{to} $\text{clique.size()} - 1$}
                \State $\text{totalWeight} \gets \text{totalWeight} + (\text{adjacencyMatrix}[\text{clique}[i]][\text{clique}[j]] > 0) + (\text{adjacencyMatrix}[\text{clique}[j]][\text{clique}[i]] > 0)$
            \EndFor
        \EndFor
        \State \textbf{return} $\text{totalWeight}$
    \EndProcedure
    \\
    \Procedure{CliqueWithMaxSize}{$\text{maxCliques}$}
        \State $\text{maxClique} \gets \text{maxCliques}[0]$
        
        \For{$\text{clique} \gets \text{maxCliques}[1, \ldots \text{maxCliques.size()}-1]$}
            \If{$\text{TotalConnections}(\text{clique}) > \text{TotalConnections}(\text{maxClique})$}
                \State $\text{maxClique} \gets \text{clique}$
            \ElsIf{$\text{TotalConnections}(\text{clique}) == \text{TotalConnections}(\text{maxClique})$}
                \If{$\text{EdgeCount}(\text{clique}) < \text{EdgeCount}(\text{maxClique})$}
                    \State $\text{maxClique} \gets \text{clique}$
                \EndIf
            \EndIf
        \EndFor
        
        \State \textbf{return} $\text{maxClique}$
    \EndProcedure

    \end{algorithmic}
\end{algorithm}

\cleardoublepage
\printbibliography[
	heading=bibintoc,
	title={Bibliography}
	]
\end{document}
